\documentclass{beamer}
\usepackage{amsmath}
\usepackage{amssymb}
\usepackage{changepage}  % Add this for adjustwidth

\title{The Fundamental Theorem of Calculus}
\author{Gary Hobson}
\date{\today}

\begin{document}

\frame{\titlepage}

% *******************************************************************
% ********************    Slide 2: Introduction    ******************
% *******************************************************************

\begin{frame}{Introduction}
\begin{itemize}
    \item The Fundamental Theorem of Calculus (FTC) establishes the relationship between differentiation and integration.
    \item It has two main parts:
    \begin{enumerate}
        \item The first part states that an antiderivative can be obtained through integration.
        \item The second part states that differentiation and integration are inverse processes.
    \end{enumerate}
\end{itemize}
\end{frame}

% *******************************************************************
% ********** Slide 3: Proof of Part 1 - Derivative Definition *******
% *******************************************************************

\begin{frame}{Proof of Part 1 - Derivative Definition}
\textbf{\normalsize Proof:}

\begin{itemize}
    \item Applying the Definition of the Derivative:\\
    To prove this formally, we use the definition of the derivative:
    \[
    F'(x) = \lim\limits_{h \to 0} \frac{F(x+h) - F(x)}{h}.
    \]

    This definition expresses the rate of change of \( F(x) \) as the limit of the difference quotient.
\end{itemize}
\end{frame}

% *******************************************************************
% ******* Slide 4: Proof of Part 1 - Derivative Definition pt2 ******
% *******************************************************************

\begin{frame}{Expressing \( F(x+h) - F(x) \) Using the Integral Definition:}

    \textbf{Since \( F(x) \) is defined as:}
    \tiny
    \[
    F(x) = \int_a^x f(t) \, dt,
    \]

  

\begin{adjustwidth}{5em}{0em}

    \textbullet\ Substituting \( x+h \) gives:
    
    \[
    F(x+h) = \int_a^{x+h} f(t) \, dt.
    \]

    \textbullet\ Now, computing the difference:

    \[
    F(x+h) - F(x) = \int_a^{x+h} f(t) \, dt - \int_a^x f(t) \, dt.
    \]

    \textbullet\ Using the additivity property of definite integrals:

    \[
    \int_a^{x+h} f(t) \, dt = \int_a^x f(t) \, dt + \int_x^{x+h} f(t) \, dt.
    \]

    \textbullet\ Thus, subtracting \( \int_a^x f(t) \, dt \) (ie.. \(F(x)\)) from both sides, we get:

    \[
    F(x+h) - F(x) = \int_x^{x+h} f(t) \, dt.
    \]
\end{adjustwidth}
\end{frame}
\newpage 

% *******************************************************************
% ********** Slide 5: Proof of Part 1 - Mean Value Theorem  *********
% *******************************************************************


\begin{frame}{Applying the Mean Value Theorem for Integrals:}

The Mean Value Theorem for Integrals states that if \( f \) is continuous on an interval \( [a, b] \), then there exists some point \( c \in [a, b] \) such that:
\begin{adjustwidth}{5em}{0em}
    \footnotesize
    \[
    \int_a^b f(t) \, dt = f(c) \cdot (b-a).
    \]
    Applying this theorem to the integral \( \int_x^{x+h} f(t) \, dt \), we conclude that there exists some \( c \in [x, x+h] \) such that:

    \[
    \int_x^{x+h} f(t) \, dt = f(c) \cdot h.
    \]

    This equation tells us that the integral over the small interval \( [x, x+h] \) can be approximated as \( f(c) \) times the width of the interval \( h \).
    \end{adjustwidth}
\end{frame}
\newpage

% *******************************************************************
% ********** Slide 6: Proof of Part 1 - Taking the Limit  ***********
% *******************************************************************
\begin{frame}{Taking the Limit:}
    \footnotesize
    Dividing both sides by \( h \), we obtain:

    \[
    \frac{F(x+h) - F(x)}{h} = f(c).
    \]

    Now, we take the limit as \( h \to 0 \). Since \( c \) is in the interval \( [x, x+h] \) and \( f(x) \) is continuous, we know that \( c \to x \) as \( h \to 0 \), so \( f(c) \to f(x) \).

    So we get:

    \[
    \lim\limits_{h \to 0} \frac{F(x+h) - F(x)}{h} = f(x).
    \]

    Since this is exactly the definition of the derivative, we conclude:

    \[
    F'(x) = f(x).
    \]
\end{frame}
\newpage

% *******************************************************************
% ********** Slide 7: Proof of Part 1 - Conclusion fo pt1 ***********
% *******************************************************************
\begin{frame}{Conclusion of Part 1:}
    \begin{itemize}
        \item This completes the proof of Part 1 of the Fundamental Theorem of Calculus.
        \item The key takeaway is that if we construct a function by integrating another function, then the derivative of that integral function brings us back to the original function.
        \item This establishes the fundamental inverse relationship between differentiation and integration.
    \end{itemize}
\end{frame}

\newpage

% *******************************************************************
% ************* Slide 8: Proof of Part 2 - The Goal *****************
% *******************************************************************
\begin{frame}{Proof of Part 2 of the Fundamental Theorem of Calculus}
%\section*{Detailed Explanation of the Proof of Part 2 of the Fundamental Theorem of Calculus (FTC)}

\textbf{Step 1: Understanding the Goal}

The second part of the Fundamental Theorem of Calculus states that:

\[
\int_{a}^{b} f(x) \,dx = F(b) - F(a),
\]

where \( F(x) \) is an antiderivative of \( f(x) \), meaning:

\[
F'(x) = f(x).
\]

This result is crucial because it tells us that we can evaluate a definite integral simply by finding an antiderivative rather than computing an infinite sum of areas.

\end{frame}
\newpage

% *******************************************************************
% ***** Slide 9: Proof of Part 2 - Partitioning the Interval ********
% *******************************************************************
\begin{frame}
\textbf{Step 2: Partitioning the Interval}

To prove this, we divide the interval \([a, b]\) into \( n \) subintervals:

\[
a = x_0 < x_1 < x_2 < \dots < x_n = b.
\]

Each subinterval has a small width \( \Delta x_i \), which is given by:

\[
\Delta x_i = x_i - x_{i-1}.
\]

The goal is to approximate the integral 

\[
\int_{a}^{b} f(x) \,dx
\]

using a Riemann sum and then take the limit as the partition gets finer.

\end{frame}
\newpage
% *******************************************************************
% ******* Slide 10: Proof of Part 2 - Mean Value Theorem ************
% *******************************************************************

\begin{frame}
\textbf{Step 3: Applying the Mean Value Theorem}

Since \( F(x) \) is differentiable, we apply the Mean Value Theorem (MVT) to each subinterval \([x_{i-1}, x_i]\). The MVT states that for each subinterval, there exists some point \( c_i \in [x_{i-1}, x_i] \) such that:

\[
F(x_i) - F(x_{i-1}) = F'(c_i) \cdot \Delta x_i.
\]

Since we know that \( F'(x) = f(x) \), this simplifies to:

\[
F(x_i) - F(x_{i-1}) = f(c_i) \cdot \Delta x_i.
\]

This equation expresses the small change in \( F(x) \) over each subinterval in terms of the function \( f(x) \).
\end{frame}
\newpage

% *******************************************************************
% ******* Slide 11: Proof of Part 2 - Summing the Intervals *********
% *******************************************************************
\begin{frame}
\textbf{Step 4: Summing Over All Subintervals}

Now, we sum this equation over all subintervals:

\[
\sum_{i=1}^{n} \left[ F(x_i) - F(x_{i-1}) \right] = \sum_{i=1}^{n} f(c_i) \Delta x_i.
\]

On the left-hand side, notice that it forms a telescoping sum:

\[
[F(x_1) - F(x_0)] + [F(x_2) - F(x_1)] + \dots + [F(x_n) - F(x_{n-1})].
\]

Since all the intermediate terms cancel out, we are left with:

\[
F(b) - F(a) = \sum_{i=1}^{n} f(c_i) \Delta x_i.
\]

The right-hand side is a Riemann sum, which approximates the integral:

\[
\int_{a}^{b} f(x) \, dx.
\]
\end{frame}
\newpage

% *******************************************************************
% ********* Slide 12: Proof of Part 2 - Taking the Limit ************
% *******************************************************************
\begin{frame}
\textbf{Step 5: Taking the Limit}

To obtain the exact value of the integral, we take the limit as the partition gets infinitely fine (i.e., as the maximum subinterval width approaches zero):

\[
\lim_{n \to \infty} \sum_{i=1}^{n} f(c_i) \Delta x_i = \int_{a}^{b} f(x) \, dx.
\]

Since the left-hand side is already equal to \( F(b) - F(a) \), we conclude:

\[
\int_{a}^{b} f(x) \, dx = F(b) - F(a).
\]

This proves that the definite integral can be computed simply by evaluating the antiderivative at the endpoints.
\end{frame}
\newpage

% *******************************************************************
% ******* Slide 13: Proof of Part 2 - Summing the Intervals *********
% *******************************************************************

\begin{frame}{Part 2 Conclusion}
    \begin{itemize}
        \item This proof shows that instead of manually computing an infinite sum of areas, we can evaluate a definite integral just by finding an antiderivative and subtracting its values at the limits of integration.
        \item The Fundamental Theorem of Calculus (Part 2) is powerful because it transforms the problem of summing an infinite number of tiny areas into a simple subtraction problem.
    \end{itemize}
\end{frame}
\newpage


% *******************************************************************
% **************** Slide 14: Applications of FTC ********************
% *******************************************************************


\begin{frame}{Applications of the Fundamental Theorem of Calculus}

    \begin{itemize}
        \tiny
        \item \textbf{Simplifies Computation:} The FTC allows us to compute definite integrals using antiderivatives, avoiding the need for summing infinite small areas. Such as:
        \[
        \int_0^1 2x \, dx = x^2 \Big|_0^1 = 1 - 0 = 1.
        \]

        \item \textbf{Physics \& Motion:} Used to calculate displacement, velocity, and acceleration in kinematics.
        \begin{itemize}
            \tiny
            \item Example: If a particle moves with velocity \( v(t) \), its displacement over time \( [a, b] \) is given by:
            \[
            \int_a^b v(t) \, dt = s(b) - s(a)
            \]
            where \( s(t) \) is the position function.
        \end{itemize}
    
        \item \textbf{Engineering \& Signal Processing:} Used in control systems, fluid dynamics, and electrical circuits.
        \begin{itemize}
            \tiny
            \item Example: Finding the charge \( Q \) stored in a capacitor by integrating the current over time:
            \[
            Q = \int_0^T I(t) \, dt
            \]
        \end{itemize}
    
        \item \textbf{Economics \& Finance:} Used to model accumulated profit, cost, and consumer/producer surplus.
        \begin{itemize}
            \tiny
            \item Example: If \( M(x) \) is the marginal cost function, the total cost over an interval is:
            \[
            \int_{x_1}^{x_2} M(x) \, dx = C(x_2) - C(x_1)
            \]
        \end{itemize}
    
        \item \textbf{Probability \& Statistics:} Helps compute probabilities from probability density functions (PDFs).
        \begin{itemize}
            \tiny
            \item Example: The probability that a continuous random variable \( X \) lies in an interval \( [a, b] \) is:
            \[
            P(a \leq X \leq b) = \int_a^b f(x) \, dx
            \]
            where \( f(x) \) is the probability density function.
        \end{itemize}
    
    \end{itemize}
    
    \end{frame}

\newpage



% *******************************************************************
% ***************** Slide 15: Conclusions of FTC ********************
% *******************************************************************



\begin{frame}{Conclusion}

    \begin{itemize}
        \footnotesize
        \item \textbf{Bridge Between Differentiation and Integration:} The Fundamental Theorem of Calculus (FTC) establishes the deep connection between these two fundamental operations.
        
        \item \textbf{Part 1 - Differentiation of Integrals:} States that the derivative of an integral of a function recovers the original function, confirming that differentiation undoes integration.
        
        \item \textbf{Part 2 - Evaluating Definite Integrals:} Provides an efficient method for computing definite integrals by evaluating an antiderivative at the limits of integration, avoiding the complexity of Riemann sums.
        
        \item \textbf{Practical Significance:} The FTC simplifies complex computations and has widespread applications in physics, engineering, economics, and probability theory.
    
        \item \textbf{Core Principle of Calculus:} Serves as a foundational result in real analysis and advanced mathematics, enabling the development of further concepts such as differential equations, multivariable calculus, and integral transformations.
    
        \item \textbf{Conclusion:} The FTC is not only a fundamental result in pure mathematics but also a powerful tool that simplifies problem-solving across various scientific and engineering disciplines.
    \end{itemize}
    
    \end{frame}

\end{document}
