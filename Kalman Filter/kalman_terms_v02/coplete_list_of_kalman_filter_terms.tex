\documentclass{article}
\usepackage{amsmath, amssymb, graphicx}
\usepackage[a4paper, margin=0.5cm]{geometry} % Adjust the margin size
\usepackage{pifont} % Add this in your preamble
\usepackage{fancyhdr} % Allows custom headers and footers
\pagestyle{fancy} % Enables custom page style
\fancyhf{} % Clears default header/footer settings
\fancyfoot[C]{\thepage} % Places page number in the center of the footer


\title{Understanding the Symbols in the Kalman Filter}
\author{}
\date{}

\begin{document}

\maketitle


\section{State Vector ($x_k$)}
\textbf{Symbol:} $x_k \in \mathbb{R}^{n}$, where $n$ is the number of state variables.  \\
\textbf{What it Represents:} The state of the system at time $k$.  \\
\textbf{Role in Kalman Filter:} The quantity we estimate at each time step.  \\
\textbf{Example:} If tracking an object's position and velocity in 1D:
\[
    x_k = \begin{bmatrix} \text{position} \\ \text{velocity} \end{bmatrix} \in \mathbb{R}^{2}
\]


\begin{table}[h]
    \centering
    \begin{tabular}{|c|c|}
        \hline
        \textbf{Term} & \textbf{Description} \\
        \hline
        \textbf{True Position} & $x_k$ \\
        & The real (but unknown) position of the car. \\
        \hline
        \textbf{Predicted Position} & $\hat{x}_k^-$ \\
        & Where we think the car should be before seeing the GPS reading. \\
        \hline
        \textbf{Measured Position} & $y_k$ \\
        & The noisy GPS measurement. \\
        \hline
        \textbf{Updated Position} & $\hat{x}_k$ \\
        & The corrected estimate after blending prediction \& measurement. \\
        \hline
    \end{tabular}
    \caption{Key Terms in Kalman Filtering for Position Estimation}
    \label{tab:kalman_terms}
\end{table}

\newpage

%*************************************************************************************************
%*************************************************************************************************
%*************************************************************************************************
%*************************************************************************************************


\section{State Transition Matrix ($A_{k-1}$)}
\textbf{Symbol:} $A_{k-1} \in \mathbb{R}^{n \times n}$  \\
\textbf{What it Represents:} Defines how the state evolves from time step $k-1$ to $k$.  \\
\textbf{Example:} If tracking position and velocity:
\[
    A_k = \begin{bmatrix} 1 & \Delta t \\ 0 & 1 \end{bmatrix}
\]  


The System Transition Matrix ($A$) is a fundamental component of the Kalman Filter, governing how the state of the system evolves over time.

\subsection*{Definition}

$A$ (also called the State Transition Matrix) describes how the state vector transforms from one time step to the next in the absence of control inputs or external influences.

Mathematically, the predicted state at time $k$ is given by:

\begin{equation}
    x_k^- = A x_{k-1} + B u_k + w_k
\end{equation}

where:

- $x_k^-$ is the predicted state at time $k$,
- $x_{k-1}$ is the previous state,
- $B u_k$ represents the control input effect,
- $w_k$ is the process noise (random disturbances).

\subsection*{Purpose and Role}

The role of $A$ is to capture the natural dynamics of the system, ensuring that:

- The state evolves correctly over time based on known physical laws.
- The previous state is propagated forward even before incorporating any new sensor measurements.
- It can represent motion models, such as constant velocity or acceleration.

\subsection*{Example: Motion in One Dimension}

For a system tracking position and velocity in 1D motion, the state vector is:

\begin{equation}
    x_k = \begin{bmatrix} 
        \text{position} \\
        \text{velocity} 
    \end{bmatrix}
\end{equation}

If the object follows constant velocity motion, the transition matrix is:

\begin{equation}
    A_k = \begin{bmatrix} 
        1 & \Delta t \\
        0 & 1 
    \end{bmatrix} = \begin{bmatrix} 
        A_{11} & A_{12} \\
        A_{21} & A_{22}
    \end{bmatrix}
\end{equation}

where:

- $1$: This coefficient ensures that the position from the previous state is carried over to the next state without any modification.\\
- $\Delta t$: This coefficient accounts for the contribution of velocity to position over the time step $\Delta t$, meaning that velocity affects position change.\\
- $0$: This coefficient indicates that position does not directly influence velocity in the transition model.\\
- $1$: This coefficient ensures that velocity remains unchanged if there is no external influence.\\

\subsection*{Key Properties of $A$}

- \textbf{Size:} $A$ is an $n \times n$ matrix, where $n$ is the number of state variables.\\
- \textbf{Identity Components:} Often contains ones along the diagonal to maintain previous state values.\\
- \textbf{Time Dependence:} $A$ can be constant or change over time for nonlinear systems.\\

\subsection*{Summary}

- The System Transition Matrix defines how the state evolves over time.\\
- It depends on the motion model of the system (e.g., constant velocity, acceleration).\\
- It is used in the prediction step of the Kalman Filter.\\
- Correctly designing $A$ ensures accurate state predictions before sensor corrections are applied.\\

\newpage

%*************************************************************************************************
%*************************************************************************************************
%*************************************************************************************************
%*************************************************************************************************


\section{Control Input Matrix (\(\mathbf{B_{k-1}}\))}

\subsection*{Definition}
The Control Input Matrix \(\mathbf{B_{k-1}}\) maps external control inputs (e.g., acceleration, force) into the state space.  
It determines how much the control input affects each state variable over time.

\subsection*{Symbol \& Dimensions}
\[
B_{k-1} \in \mathbb{R}^{n \times m}
\]
where:
\begin{itemize}
    \item \( n \) = number of state variables (e.g., position, velocity)
    \item \( m \) = number of control inputs (e.g., acceleration, steering)
\end{itemize}

\subsection*{Example: Acceleration as a Control Input}
If the system tracks position and velocity in 1D with an acceleration input, then:
\[
B_k =
\begin{bmatrix}
\frac{1}{2} \Delta t^2 \\
\Delta t
\end{bmatrix}
\]
where:
\begin{itemize}
    \item \( \frac{1}{2} \Delta t^2 \) → Controls how acceleration affects position.
    \item \( \Delta t \) → Controls how acceleration affects velocity.
\end{itemize}

\subsection*{How \(\mathbf{B}\) Works in the Kalman Filter}
\(\mathbf{B}\) appears in the state prediction equation:
\[
\hat{x}_k^- = A \hat{x}_{k-1} + B u_k
\]
where:
\begin{itemize}
    \item \( \hat{x}_k^- \) = predicted state vector
    \item \( A \) = system transition matrix
    \item \( B \) = control input matrix
    \item \( u_k \) = control input (e.g., acceleration)
\end{itemize}

\subsection*{Intuition: Why Does \(\mathbf{B}\) Look Like This?}
\begin{itemize}
    \item The first row \( \frac{1}{2} \Delta t^2 \) represents how acceleration impacts position (from physics: 
    \[
    x = x_0 + v_0 t + \frac{1}{2} a t^2
    \]
    ).
    \item The second row \( \Delta t \) represents how acceleration changes velocity over time:  
    \[
    v = v_0 + a t
    \]
\end{itemize}

\subsection*{Real-World Example: Car Moving in 1D}
\begin{itemize}
    \item \textbf{State vector:}  
    \[
    x_k =
    \begin{bmatrix}
    \text{position} \\
    \text{velocity}
    \end{bmatrix}
    \]
    \item \textbf{Control input:}  
    \[
    u_k = \text{applied acceleration}
    \]
    \item \textbf{Prediction:}  
    If 
    \[
    B_k =
    \begin{bmatrix}
    \frac{1}{2} \Delta t^2 \\
    \Delta t
    \end{bmatrix}
    \]
    Then a control input of \( u_k = 2 \) m/s\(^2\) would update position and velocity based on acceleration.
\end{itemize}

\subsection*{Key Takeaways}
\begin{itemize}
    \item \ding{51} \(\mathbf{B}\) converts control inputs into changes in state.
    \item \ding{51} Each row affects a different part of the state vector (position, velocity, etc.).
    \item \ding{51} Physics determines the structure of \(\mathbf{B}\) (e.g., motion equations).
    \item \ding{51} Larger control input means a stronger effect on state changes.
\end{itemize}

\newpage
%*************************************************************************************************
%*************************************************************************************************
%*************************************************************************************************
%*************************************************************************************************





\end{document}
