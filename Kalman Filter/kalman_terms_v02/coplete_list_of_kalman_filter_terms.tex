\documentclass{article}
\usepackage{amsmath, amssymb, graphicx}
\usepackage[a4paper, margin=0.5cm]{geometry} % Adjust the margin size
\usepackage{pifont} % Add this in your preamble
\usepackage{fancyhdr} % Allows custom headers and footers
\usepackage{booktabs}

\pagestyle{fancy} % Enables custom page style
\fancyhf{} % Clears default header/footer settings
\fancyfoot[C]{\thepage} % Places page number in the center of the footer


\title{Understanding the Symbols in the Kalman Filter}
\author{}
\date{}

\begin{document}

\maketitle


\section{State Vector ($x_k$)}
\textbf{Symbol:} $x_k \in \mathbb{R}^{n}$, where $n$ is the number of state variables.  \\
\textbf{What it Represents:} The state of the system at time $k$.  \\
\textbf{Role in Kalman Filter:} The quantity we estimate at each time step.  \\
\textbf{Example:} If tracking an object's position and velocity in 1D:
\[
    x_k = \begin{bmatrix} \text{position} \\ \text{velocity} \end{bmatrix} \in \mathbb{R}^{2}
\]


\begin{table}[h]
    \centering
    \begin{tabular}{|c|c|}
        \hline
        \textbf{Term} & \textbf{Description} \\
        \hline
        \textbf{True Position} & $x_k$ \\
        & The real (but unknown) position of the car. \\
        \hline
        \textbf{Predicted Position} & $\hat{x}_k^-$ \\
        & Where we think the car should be before seeing the GPS reading. \\
        \hline
        \textbf{Measured Position} & $y_k$ \\
        & The noisy GPS measurement. \\
        \hline
        \textbf{Updated Position} & $\hat{x}_k$ \\
        & The corrected estimate after blending prediction \& measurement. \\
        \hline
    \end{tabular}
    \caption{Key Terms in Kalman Filtering for Position Estimation}
    \label{tab:kalman_terms}
\end{table}

\newpage

%*************************************************************************************************
%*************************************************************************************************
%*************************************************************************************************
%*************************************************************************************************


\section{State Transition Matrix ($A_{k-1}$)}
\textbf{Symbol:} $A_{k-1} \in \mathbb{R}^{n \times n}$  \\
\textbf{What it Represents:} Defines how the state evolves from time step $k-1$ to $k$.  \\
\textbf{Example:} If tracking position and velocity:
\[
    A_k = \begin{bmatrix} 1 & \Delta t \\ 0 & 1 \end{bmatrix}
\]  


The System Transition Matrix ($A$) is a fundamental component of the Kalman Filter, governing how the state of the system evolves over time.

\subsection*{Definition}

$A$ (also called the State Transition Matrix) describes how the state vector transforms from one time step to the next in the absence of control inputs or external influences.

Mathematically, the predicted state at time $k$ is given by:

\begin{equation}
    x_k^- = A x_{k-1} + B u_k + w_k
\end{equation}

where:

- $x_k^-$ is the predicted state at time $k$,
- $x_{k-1}$ is the previous state,
- $B u_k$ represents the control input effect,
- $w_k$ is the process noise (random disturbances).

\subsection*{Purpose and Role}

The role of $A$ is to capture the natural dynamics of the system, ensuring that:

- The state evolves correctly over time based on known physical laws.
- The previous state is propagated forward even before incorporating any new sensor measurements.
- It can represent motion models, such as constant velocity or acceleration.

\subsection*{Example: Motion in One Dimension}

For a system tracking position and velocity in 1D motion, the state vector is:

\begin{equation}
    x_k = \begin{bmatrix} 
        \text{position} \\
        \text{velocity} 
    \end{bmatrix}
\end{equation}

If the object follows constant velocity motion, the transition matrix is:

\begin{equation}
    A_k = \begin{bmatrix} 
        1 & \Delta t \\
        0 & 1 
    \end{bmatrix} = \begin{bmatrix} 
        A_{11} & A_{12} \\
        A_{21} & A_{22}
    \end{bmatrix}
\end{equation}

where:

- $1$: This coefficient ensures that the position from the previous state is carried over to the next state without any modification.\\
- $\Delta t$: This coefficient accounts for the contribution of velocity to position over the time step $\Delta t$, meaning that velocity affects position change.\\
- $0$: This coefficient indicates that position does not directly influence velocity in the transition model.\\
- $1$: This coefficient ensures that velocity remains unchanged if there is no external influence.\\

\subsection*{Key Properties of $A$}

- \textbf{Size:} $A$ is an $n \times n$ matrix, where $n$ is the number of state variables.\\
- \textbf{Identity Components:} Often contains ones along the diagonal to maintain previous state values.\\
- \textbf{Time Dependence:} $A$ can be constant or change over time for nonlinear systems.\\

\subsection*{Summary}

- The System Transition Matrix defines how the state evolves over time.\\
- It depends on the motion model of the system (e.g., constant velocity, acceleration).\\
- It is used in the prediction step of the Kalman Filter.\\
- Correctly designing $A$ ensures accurate state predictions before sensor corrections are applied.\\

\newpage

%*************************************************************************************************
%*************************************************************************************************
%*************************************************************************************************
%*************************************************************************************************


\section{Control Input Matrix (\(\mathbf{B_{k-1}}\))}

\subsection*{Definition}
The Control Input Matrix \(\mathbf{B_{k-1}}\) maps external control inputs (e.g., acceleration, force) into the state space.  
It determines how much the control input affects each state variable over time.

\subsection*{Symbol \& Dimensions}
\[
B_{k-1} \in \mathbb{R}^{n \times m}
\]
where:
\begin{itemize}
    \item \( n \) = number of state variables (e.g., position, velocity)
    \item \( m \) = number of control inputs (e.g., acceleration, steering)
\end{itemize}

\subsection*{Example: Acceleration as a Control Input}
If the system tracks position and velocity in 1D with an acceleration input, then:
\[
B_k =
\begin{bmatrix}
\frac{1}{2} \Delta t^2 \\
\Delta t
\end{bmatrix}
\]
where:
\begin{itemize}
    \item \( \frac{1}{2} \Delta t^2 \) → Controls how acceleration affects position.
    \item \( \Delta t \) → Controls how acceleration affects velocity.
\end{itemize}

\subsection*{How \(\mathbf{B}\) Works in the Kalman Filter}
\(\mathbf{B}\) appears in the state prediction equation:
\[
\hat{x}_k^- = A \hat{x}_{k-1} + B u_k
\]
where:
\begin{itemize}
    \item \( \hat{x}_k^- \) = predicted state vector
    \item \( A \) = system transition matrix
    \item \( B \) = control input matrix
    \item \( u_k \) = control input (e.g., acceleration)
\end{itemize}

\subsection*{Intuition: Why Does \(\mathbf{B}\) Look Like This?}
\begin{itemize}
    \item The first row \( \frac{1}{2} \Delta t^2 \) represents how acceleration impacts position (from physics: 
    \[
    x = x_0 + v_0 t + \frac{1}{2} a t^2
    \]
    ).
    \item The second row \( \Delta t \) represents how acceleration changes velocity over time:  
    \[
    v = v_0 + a t
    \]
\end{itemize}

\subsection*{Real-World Example: Car Moving in 1D}
\begin{itemize}
    \item \textbf{State vector:}  
    \[
    x_k =
    \begin{bmatrix}
    \text{position} \\
    \text{velocity}
    \end{bmatrix}
    \]
    \item \textbf{Control input:}  
    \[
    u_k = \text{applied acceleration}
    \]
    \item \textbf{Prediction:}  
    If 
    \[
    B_k =
    \begin{bmatrix}
    \frac{1}{2} \Delta t^2 \\
    \Delta t
    \end{bmatrix}
    \]
    Then a control input of \( u_k = 2 \) m/s\(^2\) would update position and velocity based on acceleration.
\end{itemize}

\subsection*{Key Takeaways}
\begin{itemize}
    \item \ding{51} \(\mathbf{B}\) converts control inputs into changes in state.
    \item \ding{51} Each row affects a different part of the state vector (position, velocity, etc.).
    \item \ding{51} Physics determines the structure of \(\mathbf{B}\) (e.g., motion equations).
    \item \ding{51} Larger control input means a stronger effect on state changes.
\end{itemize}

\newpage
%*************************************************************************************************
%*************************************************************************************************
%*************************************************************************************************
%*************************************************************************************************

\section*{External Control Input (\(\mathbf{u_k}\))}

\subsection*{Definition}
The control input vector \(\mathbf{u_k}\) represents external forces or influences that actively change the system’s state.
It is used when the system is not just evolving naturally, but is also being controlled by an external factor (e.g., acceleration, steering, applied force).

\subsection*{Symbol \& Dimensions}
\[
\mathbf{u_k} \in \mathbb{R}^m
\]
where:
\begin{itemize}
    \item \(m\) = number of control inputs.
\end{itemize}

\subsection*{Example: Acceleration as a Control Input}
If we are tracking an object’s position and velocity in one dimension (1D) and applying an acceleration, then:

\[
\mathbf{u_k} = \text{acceleration}
\]

\begin{itemize}
    \item If no acceleration is applied, \(\mathbf{u_k} = 0\).
    \item If a constant acceleration of 2 m/s² is applied, \(\mathbf{u_k} = 2\).
\end{itemize}

\subsection*{How is \(\mathbf{u_k}\) Measured or Estimated?}
The control input \(\mathbf{u_k}\) is often measured by sensors or computed based on system dynamics:

\begin{table}[h]
    \centering
    \begin{tabular}{|c|c|c|}
        \hline
        \textbf{Control Input (\(\mathbf{u_k}\))} & \textbf{Sensor Used} & \textbf{How Data is Obtained} \\
        \hline
        Acceleration (m/s²) & IMU (Inertial Measurement Unit) & Directly measured by accelerometers. \\
        Throttle Input (Vehicle Control) & ECU (Engine Control Unit) & Read from vehicle's onboard system. \\
        Steering Angle (for turning systems) & Gyroscope / Steering Angle Sensor & Measures wheel angle in degrees. \\
        Force Applied (e.g., robotic arms) & Load Cells / Force Sensors & Measures external force applied. \\
        \hline
    \end{tabular}
\end{table}

\subsection*{Extrapolating \(\mathbf{u_k}\) From Data}
If a system doesn’t have a direct sensor for \(\mathbf{u_k}\), it can be approximated from other data sources:

\begin{itemize}
    \item \textbf{Velocity Change:} If velocity measurements from GPS or wheel encoders show an increasing trend, acceleration can be estimated as:
    \[
    \mathbf{u_k} = \frac{\Delta v}{\Delta t}
    \]
    \item \textbf{Predicting Future Inputs:} In robotics, control commands (e.g., PWM signals to motors) can be used as \(\mathbf{u_k}\), assuming known system dynamics.
\end{itemize}

\subsection*{How \(\mathbf{u_k}\) Works in the Kalman Filter}
The state prediction equation includes \(\mathbf{u_k}\):

\[
\hat{x_k}^- = A \hat{x_{k-1}} + B u_k
\]

where:
\begin{itemize}
    \item \(\hat{x_k}^-\) = Predicted state (before correction).
    \item \(A\) = State transition matrix (natural system evolution).
    \item \(B\) = Control input matrix (how control input affects the state).
    \item \(u_k\) = External control input (external influence like acceleration or force).
\end{itemize}

\subsection*{Intuition: Why Does \(\mathbf{u_k}\) Matter?}
\begin{itemize}
    \item If a car is moving without acceleration, its position will change only due to velocity.
    \item If a driver presses the gas pedal, acceleration \(\mathbf{u_k}\) modifies the velocity and position.
    \item The control input \(\mathbf{u_k}\) ensures the Kalman filter accounts for these external actions.
\end{itemize}

\newpage
%*************************************************************************************************
%*************************************************************************************************
%*************************************************************************************************
%*************************************************************************************************



\section{How to Solve for \(\mathbf{P_k^-}\) (Predicted Covariance Matrix)}

The predicted covariance matrix \(\mathbf{P_k^-}\) represents how uncertain we are about our state prediction before incorporating a measurement. It is computed using the process model and process noise covariance \(\mathbf{Q_k}\).

\subsection*{Step 1: The Prediction Equation for \(\mathbf{P_k^-}\)}
\[
\mathbf{P_k^-} = \mathbf{A} \mathbf{P_{k-1}} \mathbf{A}^T + \mathbf{Q_k}
\]

Where:
\begin{itemize}
    \item \(\mathbf{P_k^-}\) = Predicted covariance matrix (uncertainty in the predicted state).
    \item \(\mathbf{A}\) = State transition matrix (how the system evolves over time).
    \item \(\mathbf{P_{k-1}}\) = Previous covariance matrix (uncertainty from the previous step).
    \item \(\mathbf{A}^T\) = Transpose of \(\mathbf{A}\) (ensures proper matrix dimensions).
    \item \(\mathbf{Q_k}\) = Process noise covariance matrix (models uncertainty in the system dynamics).
\end{itemize}

\subsection*{Step 2: Understanding Each Term}
\textbf{The Role of \(\mathbf{P_{k-1}}\):}
\begin{itemize}
    \item If \(\mathbf{P_{k-1}}\) is large, we were very uncertain about the previous state.
    \item If \(\mathbf{P_{k-1}}\) is small, we were confident in our previous estimate.
\end{itemize}

\textbf{The Role of \(\mathbf{A}\) and \(\mathbf{A^T}\):}
\begin{itemize}
    \item \(\mathbf{A}\) transforms the uncertainty as it propagates the system forward in time.
    \item Multiplying \(\mathbf{A P_{k-1} A^T}\) adjusts the uncertainty according to the system's motion model.
\end{itemize}

\textbf{The Role of \(\mathbf{Q_k}\):}
\begin{itemize}
    \item \(\mathbf{Q_k}\) increases the uncertainty, accounting for random disturbances.
    \item This prevents the filter from being overconfident in its predictions.
\end{itemize}

\subsection*{Step 3: Solving for \(\mathbf{P_k^-}\) with an Example}
Let's consider a car moving in a straight line where the state is:

\[
\mathbf{x_k} =
\begin{bmatrix}
\text{position} \\
\text{velocity}
\end{bmatrix}
\]

And the state transition matrix is:

\[
\mathbf{A} =
\begin{bmatrix}
1 & \Delta t \\
0 & 1
\end{bmatrix}
\]

Let's assume:

\[
\mathbf{P_{k-1}} =
\begin{bmatrix}
0.1 & 0 \\
0 & 0.1
\end{bmatrix}
\]

\[
\mathbf{Q_k} =
\begin{bmatrix}
0.01 & 0.02 \\
0.02 & 0.04
\end{bmatrix}
\]

Now, solve for \(\mathbf{P_k^-}\):

\[
\mathbf{P_k^-} = \mathbf{A P_{k-1} A^T} + \mathbf{Q_k}
\]

\subsection*{Step-by-Step Matrix Multiplication}
1. Compute \(\mathbf{A P_{k-1}}\):

\[
\mathbf{A P_{k-1}} =
\begin{bmatrix}
1 & \Delta t \\
0 & 1
\end{bmatrix}
\begin{bmatrix}
0.1 & 0 \\
0 & 0.1
\end{bmatrix}
=
\begin{bmatrix}
0.1 & 0.1 \Delta t \\
0 & 0.1
\end{bmatrix}
\]

2. Compute \((\mathbf{A P_{k-1}}) \mathbf{A^T}\):

\[
\begin{bmatrix}
0.1 & 0.1 \Delta t \\
0 & 0.1
\end{bmatrix}
\begin{bmatrix}
1 & 0 \\
\Delta t & 1
\end{bmatrix}
=
\begin{bmatrix}
0.1 + 0.1 \Delta t^2 & 0.1 \Delta t \\
0.1 \Delta t & 0.1
\end{bmatrix}
\]

3. Add \(\mathbf{Q_k}\) to the Result:

\[
\mathbf{P_k^-} =
\begin{bmatrix}
0.1 + 0.1 \Delta t^2 & 0.1 \Delta t \\
0.1 \Delta t & 0.1
\end{bmatrix}
+
\begin{bmatrix}
0.01 & 0.02 \\
0.02 & 0.04
\end{bmatrix}
\]

\[
=
\begin{bmatrix}
0.11 + 0.1 \Delta t^2 & 0.1 \Delta t + 0.02 \\
0.1 \Delta t + 0.02 & 0.14
\end{bmatrix}
\]

Final Result:

\[
\mathbf{P_k^-} =
\begin{bmatrix}
0.11 + 0.1 \Delta t^2 & 0.1 \Delta t + 0.02 \\
0.1 \Delta t + 0.02 & 0.14
\end{bmatrix}
\]

\subsection*{Step 4: How Does \(\mathbf{P_k^-}\) Affect the Kalman Gain?}
Once we compute \(\mathbf{P_k^-}\), it directly affects the Kalman Gain:

\[
\mathbf{K_k} =
\mathbf{P_k^-} \mathbf{H}^T
(\mathbf{H P_k^- H}^T + \mathbf{R})^{-1}
\]

\begin{itemize}
    \item If \(\mathbf{P_k^-}\) is large, the Kalman Gain \(\mathbf{K_k}\) increases, meaning the filter trusts the measurement more.
    \item If \(\mathbf{P_k^-}\) is small, the filter trusts its prediction more and ignores noisy measurements.
\end{itemize}

Thus, updating \(\mathbf{P_k^-}\) correctly is essential for the Kalman filter to balance between trusting the model and trusting sensor data.

\subsection*{Step 5: Summary}
\begin{itemize}
    \item \(\mathbf{P_k^-}\) tracks our uncertainty in the predicted state.
    \item We compute \(\mathbf{P_k^-}\) using \(\mathbf{A P_{k-1} A^T} + \mathbf{Q_k}\).
    \item \(\mathbf{Q_k}\) ensures the filter accounts for unpredictable real-world changes.
    \item A higher \(\mathbf{P_k^-}\) increases the Kalman Gain \(\mathbf{K_k}\), making the filter trust the sensor more.
\end{itemize}

\newpage
%*************************************************************************************************
%*************************************************************************************************
%*************************************************************************************************
%*************************************************************************************************

\section{Understanding the Process Noise Covariance Matrix \(\mathbf{Q_k}\)}

\subsection*{What is \(\mathbf{Q_k}\)?}
The process noise covariance matrix \(\mathbf{Q_k}\) represents the uncertainty in the system's motion model. It accounts for random effects that influence the system but are not explicitly modeled, such as:
\begin{itemize}
    \item Friction
    \item Wind resistance
    \item Sensor drift
    \item Unmodeled acceleration changes
\end{itemize}

\subsection*{What Does \(\mathbf{Q_k}\) Do in the Kalman Filter?}
\textbf{Introduces Uncertainty in Predictions}
\begin{itemize}
    \item Since real-world systems do not follow perfect mathematical models, \(\mathbf{Q_k}\) allows the filter to remain flexible and not overly confident in its predictions.
\end{itemize}

\textbf{Prevents the Filter from Ignoring Real-World Changes}
\begin{itemize}
    \item If \(\mathbf{Q_k}\) is too small, the filter assumes the model is perfect and does not adjust when unexpected changes occur.
    \item If \(\mathbf{Q_k}\) is too large, the filter relies too much on sensor data, leading to overcorrections.
\end{itemize}

\textbf{Affects the Kalman Gain \(\mathbf{K_k}\)}
\begin{itemize}
    \item A higher \(\mathbf{Q_k}\) results in higher uncertainty in the prediction, making the filter trust sensor measurements more.
    \item A lower \(\mathbf{Q_k}\) makes the filter trust its motion model more.
\end{itemize}

\subsection*{Where Can You Find Data for \(\mathbf{Q_k}\)?}
\(\mathbf{Q_k}\) is usually estimated using known system properties, experimental data, or manufacturer specifications.

\textbf{From Sensor Data Sheets}
\begin{itemize}
    \item If you are using a GPS like the NEO-M8U, the datasheet provides velocity accuracy and position drift information, which can be used to estimate \(\mathbf{Q_k}\).
    \item Example: If the GPS reports position uncertainty of 1.5 m and velocity uncertainty of 0.05 m/s, these values help determine reasonable process noise values.
\end{itemize}

\textbf{By Experimentation}
\begin{itemize}
    \item Collect real-world data and measure how much the system deviates from expected motion over time.
    \item Compute the variance of these deviations to approximate \(\mathbf{Q_k}\).
\end{itemize}

\textbf{By Analytical Estimation}
\begin{itemize}
    \item If the system has known physical properties (e.g., a robot moving with known acceleration noise), we can compute \(\mathbf{Q_k}\) based on physics-based models.
\end{itemize}

\subsection*{Example of \(\mathbf{Q_k}\) for a Car Moving in a Straight Line}
For a car tracking position and velocity, a reasonable choice for \(\mathbf{Q_k}\) might be:

\[
\mathbf{Q_k} =
\begin{bmatrix}
0.0125 & 0.025 \\
0.025 & 0.05
\end{bmatrix}
\]

\begin{itemize}
    \item \(0.0125\) → Small position variance (position uncertainty grows slowly over time).
    \item \(0.05\) → Larger velocity variance (since acceleration may change more unpredictably).
\end{itemize}

These values prevent the filter from being overconfident while still keeping predictions stable.

\subsection*{Summary of \(\mathbf{Q_k}\)}
\begin{itemize}
    \item \(\mathbf{Q_k}\) represents uncertainty in the system’s motion model.
    \item It accounts for random effects like sensor drift, friction, or external forces.
    \item \(\mathbf{Q_k}\) balances how much the filter trusts its motion model vs. sensor data.
    \item It is estimated from datasheets, experiments, or physics-based models.
\end{itemize}

\newpage
%*************************************************************************************************
%*************************************************************************************************
%*************************************************************************************************
%*************************************************************************************************

\section*{Understanding the Measurement Matrix \(\mathbf{H_k}\)}

\subsection{What is \(\mathbf{H_k}\)?}
The measurement matrix \(\mathbf{H_k}\) defines how the system’s internal state is mapped to the measurements from a sensor. Since the Kalman filter tracks an entire state vector, but most sensors only measure part of that state, \(\mathbf{H_k}\) acts as a bridge between the state estimate and what is actually observed.

\subsection*{What Does \(\mathbf{H_k}\) Do in the Kalman Filter?}
\textbf{Extracts Measured Variables from the State Vector}
\begin{itemize}
    \item Not all elements in the state vector are directly observable.
    \item \(\mathbf{H_k}\) ensures that the filter only compares predictions for measurable quantities with real sensor data.
\end{itemize}

\textbf{Transforms the State to Match Sensor Data}
\begin{itemize}
    \item The Kalman filter internally tracks a state vector like \(\begin{bmatrix} \text{position} \\ \text{velocity} \end{bmatrix}\), but a GPS sensor may only provide position.
    \item \(\mathbf{H_k}\) is used to extract only the position from the state vector so it can be compared with the GPS measurement.
\end{itemize}

\textbf{Affects the Kalman Gain \(\mathbf{K_k}\)}
\begin{itemize}
    \item \(\mathbf{H_k}\) helps the filter determine how much to adjust the estimate based on how reliable and useful the measurement is.
    \item If \(\mathbf{H_k}\) is incorrect or missing key state information, the Kalman filter won't update properly.
\end{itemize}

\subsection*{Where Can You Find \(\mathbf{H_k}\)?}
\(\mathbf{H_k}\) is determined by the type of sensor used.

\textbf{From Sensor Specifications}
\begin{itemize}
    \item If the sensor only measures position, \(\mathbf{H_k}\) selects only the position component from the state vector.
    \item If the sensor measures both position and velocity, \(\mathbf{H_k}\) is larger and selects both components.
\end{itemize}

\textbf{By Knowing What Data is Available}
\begin{itemize}
    \item If using a GPS, it provides position, so \(\mathbf{H_k}\) should extract the position component.
    \item If using an IMU (accelerometer + gyroscope), it provides velocity and acceleration, so \(\mathbf{H_k}\) must extract those values.
\end{itemize}

\subsection*{Example: Car Moving in a Straight Line}
Assume the state vector tracks position and velocity:
\[
\mathbf{x_k} =
\begin{bmatrix}
\text{position} \\
\text{velocity}
\end{bmatrix}
\]

If a GPS sensor only measures position, then:
\[
\mathbf{H_k} =
\begin{bmatrix}
1 & 0
\end{bmatrix}
\]
This means:
\begin{itemize}
    \item The \(1\) extracts the position from the state vector.
    \item The \(0\) ignores velocity since GPS does not measure it.
\end{itemize}

If the sensor measured both position and velocity, then:
\[
\mathbf{H_k} =
\begin{bmatrix}
1 & 0 \\
0 & 1
\end{bmatrix}
\]
This tells the Kalman filter to use both position and velocity measurements.

\subsection*{Summary of \(\mathbf{H_k}\)}
\begin{itemize}
    \item \(\mathbf{H_k}\) defines how the state is mapped to sensor measurements.
    \item It extracts only the measured parts of the state vector.
    \item \(\mathbf{H_k}\) must be chosen based on the type of sensor used.
    \item Incorrect \(\mathbf{H_k}\) leads to incorrect state updates.
\end{itemize}


\newpage
%*************************************************************************************************
%*************************************************************************************************
%*************************************************************************************************
%*************************************************************************************************

\section{Understanding the Measurement Noise Covariance Matrix \(\mathbf{R_k}\)}

\subsection*{What is \(\mathbf{R_k}\)?}
The measurement noise covariance matrix \(\mathbf{R_k}\) represents the uncertainty in sensor measurements. It defines how much trust the Kalman filter should place in the sensor data by quantifying the random noise introduced by the sensor.

\subsection*{What Does \(\mathbf{R_k}\) Do in the Kalman Filter?}

\textbf{Models Sensor Accuracy}
\begin{itemize}
    \item Every sensor has some amount of error (e.g., GPS error in meters, accelerometer drift, or radar noise).
    \item \(\mathbf{R_k}\) ensures the Kalman filter accounts for this noise, preventing it from overreacting to bad measurements.
\end{itemize}

\textbf{Determines Trust in Sensor Data}
\begin{itemize}
    \item If \(\mathbf{R_k}\) is small, the filter trusts sensor data more and makes strong corrections based on measurements.
    \item If \(\mathbf{R_k}\) is large, the filter trusts the motion model more and ignores noisy sensor readings.
\end{itemize}

\textbf{Affects the Kalman Gain \(\mathbf{K_k}\)}
\begin{itemize}
    \item Higher \(\mathbf{R_k} \Rightarrow\) Lower Kalman Gain \(\mathbf{K_k}\) \(\Rightarrow\) More reliance on the model, less on the measurement.
    \item Lower \(\mathbf{R_k} \Rightarrow\) Higher Kalman Gain \(\mathbf{K_k}\) \(\Rightarrow\) More reliance on the sensor data.
\end{itemize}

\subsection*{Where Can You Find \(\mathbf{R_k}\)?}
\(\mathbf{R_k}\) is usually determined from sensor specifications or experimentally estimated.

\textbf{From Sensor Data Sheets}
\begin{itemize}
    \item A GPS datasheet might say "position accuracy \(\pm 1.5\) meters."
    \item This translates to: 
    \[
    \mathbf{R_k} = 1.5^2 = 2.25
    \]
    \item A radar sensor might have an angular measurement error of \(\pm 0.5\) degrees, leading to:
    \[
    \mathbf{R_k} = 0.5^2 = 0.25
    \]
\end{itemize}

\textbf{By Experimentation}
\begin{itemize}
    \item Collect real-world sensor data and compare it to ground truth.
    \item Compute the variance of the measurement errors over time to determine \(\mathbf{R_k}\).
\end{itemize}

\textbf{By Manufacturer Calibration}
\begin{itemize}
    \item Some sensors provide factory-calibrated covariance values that can be used directly.
\end{itemize}

\subsection*{Example: Car Moving in a Straight Line with GPS}
Assume a GPS sensor measures position with a standard deviation of \(1.5\) meters:
\[
\mathbf{R_k} =
\begin{bmatrix}
2.25
\end{bmatrix}
\]

If the GPS also measures velocity with \(0.2\) m/s standard deviation, then:
\[
\mathbf{R_k} =
\begin{bmatrix}
2.25 & 0 \\
0 & 0.04
\end{bmatrix}
\]

This means:
\begin{itemize}
    \item The position measurement has higher uncertainty (\(2.25\) variance).
    \item The velocity measurement is more precise (\(0.04\) variance).
\end{itemize}

\newpage

\section*{Difference Between \(\mathbf{R_k}\) and \(\mathbf{Q_k}\) in the Kalman Filter}

\(\mathbf{R_k}\) and \(\mathbf{Q_k}\) both represent uncertainty in the Kalman filter, but they apply to different parts of the system.

\subsection*{1. Process Noise Covariance \(\mathbf{Q_k}\)}

\textbf{What It Represents:}  
\(\mathbf{Q_k}\) models uncertainty in the motion model, accounting for random forces or unmodeled dynamics that affect the system state.

\textbf{Where It Appears:}  
Used in the prediction step to update the uncertainty in the predicted state:
\[
\mathbf{P_k^-} = \mathbf{A} \mathbf{P_{k-1}} \mathbf{A}^T + \mathbf{Q_k}
\]

\textbf{What Causes It:}
\begin{itemize}
    \item Small unmodeled accelerations (e.g., wind, road bumps in a car, drift in an IMU).
    \item Imperfect system dynamics (e.g., friction that is not modeled).
    \item Variability in control inputs (e.g., driver inconsistency in applying acceleration).
\end{itemize}

\textbf{Where It Comes From:}
\begin{itemize}
    \item Estimated from physical properties of the system (e.g., acceleration noise).
    \item Can be experimentally measured by analyzing how much real-world data deviates from an ideal motion model.
    \item Sometimes manually tuned to improve filter performance.
\end{itemize}

\subsection*{2. Measurement Noise Covariance \(\mathbf{R_k}\)}

\textbf{What It Represents:}  
\(\mathbf{R_k}\) models uncertainty in the sensor readings, accounting for inaccuracies in measurements.

\textbf{Where It Appears:}  
Used in the correction step when incorporating a new sensor measurement:
\[
\mathbf{K_k} = \mathbf{P_k^-} \mathbf{H}^T (\mathbf{H} \mathbf{P_k^-} \mathbf{H}^T + \mathbf{R_k})^{-1}
\]

\textbf{What Causes It:}
\begin{itemize}
    \item Sensor precision limits (e.g., a GPS might have \(\pm1.5\) meters of error).
    \item Environmental conditions (e.g., noise in radar due to weather).
    \item Electronic noise in the sensor circuitry.
\end{itemize}

\textbf{Where It Comes From:}
\begin{itemize}
    \item Sensor datasheets (e.g., GPS accuracy reports, IMU noise levels).
    \item Measured from real data by calculating the variance of sensor errors.
    \item Factory calibration provided by the manufacturer.
\end{itemize}

\subsection*{Key Differences Between \(\mathbf{Q_k}\) and \(\mathbf{R_k}\)}

\begin{table}[h]
    \centering
    \renewcommand{\arraystretch}{1.3}
    \begin{tabular}{@{}lcc@{}}
        \toprule
        \textbf{Feature} & \(\mathbf{Q_k}\) (Process Noise Covariance) & \(\mathbf{R_k}\) (Measurement Noise Covariance) \\
        \midrule
        What It Affects & State transition (motion model uncertainty) & Sensor readings (measurement noise) \\
        Used In & Prediction step & Update step (correction) \\
        Caused By & External forces, unmodeled system effects & Sensor inaccuracies, electronic noise \\
        Where It Comes From & Estimated from physics, experiments, tuning & Sensor datasheets, calibration, experiments \\
        Effect if Too Large & Filter becomes too flexible, ignores the motion model & Filter ignores sensor data, relies too much on the model \\
        Effect if Too Small & Filter becomes overconfident in motion model & Filter trusts noisy measurements too much \\
        \bottomrule
    \end{tabular}
\end{table}

\subsection*{Example: Car Driving with GPS}

Let’s say we are tracking a car’s position and velocity using a GPS sensor.

\(\mathbf{Q_k}\) accounts for uncertainties in motion (e.g., bumps in the road, wind, or unknown acceleration).
\[
\mathbf{Q_k} =
\begin{bmatrix}
0.0125 & 0.025 \\
0.025  & 0.05
\end{bmatrix}
\]
This means we expect small acceleration changes over time.

\(\mathbf{R_k}\) accounts for measurement noise in the GPS sensor (e.g., signal interference or GPS drift).
\[
\mathbf{R_k} =
\begin{bmatrix}
2.25
\end{bmatrix}
\]
This means the GPS has a variance of 2.25 meters in its position readings.


\newpage
%*************************************************************************************************
%*************************************************************************************************
%*************************************************************************************************
%*************************************************************************************************


\section{Understanding the Measurement Noise Vector \(\mathbf{v_k}\)}

\subsection*{What is \(\mathbf{v_k}\)?}
The measurement noise vector \(\mathbf{v_k}\) represents the random errors in sensor measurements at each time step. These errors occur because real-world sensors are never perfect and always introduce some level of uncertainty or noise into the readings.

\subsection*{What Does \(\mathbf{v_k}\) Do in the Kalman Filter?}

\textbf{Accounts for Sensor Errors}  
\begin{itemize}
    \item No sensor provides a perfectly accurate measurement. \(\mathbf{v_k}\) models random fluctuations in sensor data due to noise.
\end{itemize}

\textbf{Prevents Overconfidence in Measurements}  
\begin{itemize}
    \item Without \(\mathbf{v_k}\), the Kalman filter would assume the sensor data is perfect, which is never true in reality.
    \item By considering \(\mathbf{v_k}\), the filter blends sensor readings with the predicted state to improve accuracy.
\end{itemize}

\textbf{Influences the Kalman Gain \(\mathbf{K_k}\)}  
\begin{itemize}
    \item If \(\mathbf{v_k}\) is large, the filter trusts the motion model more and updates less based on measurements.
    \item If \(\mathbf{v_k}\) is small, the filter trusts the sensor data more and relies heavily on new measurements.
\end{itemize}

\subsection*{Where Does \(\mathbf{v_k}\) Come From?}

\textbf{Sensor Specifications}  
\begin{itemize}
    \item The sensor’s datasheet typically provides the expected noise level.
    \item Example: A GPS might have an error of \(\pm1.5\) meters, which means \(\mathbf{v_k}\) follows a normal distribution with that variance.
\end{itemize}

\textbf{Experimental Measurement}  
\begin{itemize}
    \item Collect multiple readings from a sensor in a controlled environment and compare them to the ground truth.
    \item Compute the variance of the errors to estimate \(\mathbf{v_k}\).
\end{itemize}

\textbf{Randomly Generated in Simulations}  
\begin{itemize}
    \item When simulating a Kalman filter, \(\mathbf{v_k}\) is often generated as random noise from a normal distribution:
    \[
    \mathbf{v_k} \sim \mathcal{N}(0, \mathbf{R_k})
    \]
    \item This ensures the simulation accounts for realistic sensor errors.
\end{itemize}

\subsection*{Example: Car Tracking with GPS}

Imagine tracking a car's position using GPS. The state vector tracks position and velocity:
\[
\mathbf{x_k} =
\begin{bmatrix}
\text{position} \\
\text{velocity}
\end{bmatrix}
\]
The GPS only measures position, but with some error.

Suppose the true position is 50 meters, but the GPS reports 50.8 meters due to noise.

The measurement noise vector is:
\[
\mathbf{v_k} = \text{Measured Position} - \text{True Position} = 50.8 - 50 = 0.8
\]
Here, \(\mathbf{v_k} = 0.8\) meters, meaning the GPS overestimated the position by 0.8 meters.

If we assume \(\mathbf{v_k}\) follows a normal distribution with zero mean and variance from \(\mathbf{R_k}\), we might say:
\[
\mathbf{v_k} \sim \mathcal{N}(0, 2.25)
\]
This means GPS errors are normally distributed around zero with a variance of 2.25 meters.

\subsection*{Summary of \(\mathbf{v_k}\)}
\begin{itemize}
    \item Represents random noise in sensor measurements.
    \item Accounts for inaccuracies in real-world sensors.
    \item Estimated from sensor specs, experiments, or random noise in simulations.
    \item Works with \(\mathbf{R_k}\) to determine how much trust to place in sensor data.
\end{itemize}

\newpage
%*************************************************************************************************
%*************************************************************************************************
%*************************************************************************************************
%*************************************************************************************************








\end{document}
