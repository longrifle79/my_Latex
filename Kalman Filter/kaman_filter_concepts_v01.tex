\documentclass{article}
\usepackage{amsmath, amssymb, graphicx}

\title{Understanding the Symbols in the Kalman Filter}
\author{}
\date{}

\begin{document}

\maketitle

\section{Discrete-Time Instant ($k$)}
\textbf{Symbol:} $k$  \\
\textbf{What it Represents:} A discrete time step in the system.  \\
\textbf{Role in Kalman Filter:} The Kalman filter operates recursively, updating the state estimate at each time step $k$.  \\
\textbf{Example:} If you are tracking an object moving every 0.1s:
\begin{itemize}
    \item $k = 0$ (initial time step)
    \item $k = 1$ (after 0.1s)
    \item $k = 2$ (after 0.2s), etc.
\end{itemize}

\section{State Vector ($x_k$)}
\textbf{Symbol:} $x_k \in \mathbb{R}^{n}$, where $n$ is the number of state variables.  \\
\textbf{What it Represents:} The state of the system at time $k$.  \\
\textbf{Role in Kalman Filter:} The quantity we estimate at each time step.  \\
\textbf{Example:} If tracking an object's position and velocity in 1D:
\[
    x_k = \begin{bmatrix} \text{position} \\ \text{velocity} \end{bmatrix} \in \mathbb{R}^{2}
\]

\section{Control Input Vector ($u_{k-1}$)}
\textbf{Symbol:} $u_{k-1} \in \mathbb{R}^{m}$, where $m$ is the number of control inputs.  \\
\textbf{What it Represents:} The external control inputs applied at time $k-1$.  \\
\textbf{Example:} If an external acceleration affects an object's motion:
\[
    u_{k-1} = \text{applied acceleration}
\]

\section{Process Noise Vector ($w_{k-1}$)}
\textbf{Symbol:} $w_{k-1} \in \mathbb{R}^{n}$  \\
\textbf{What it Represents:} Random disturbances in the system model (process noise).  \\
\textbf{Role in Kalman Filter:} Accounts for uncertainty in the system dynamics.  \\
\textbf{Example:} Wind force affecting an object's motion.  \\
\textbf{Assumptions:} White noise, zero mean, and covariance matrix:
\[
    Q_k = E[w_k w_k^T]
\]

\section{State Transition Matrix ($A_{k-1}$)}
\textbf{Symbol:} $A_{k-1} \in \mathbb{R}^{n \times n}$  \\
\textbf{What it Represents:} Defines how the state evolves from time step $k-1$ to $k$.  \\
\textbf{Example:} If tracking position and velocity:
\[
    A_k = \begin{bmatrix} 1 & \Delta t \\ 0 & 1 \end{bmatrix}
\]

\section{Control Input Matrix ($B_{k-1}$)}
\textbf{Symbol:} $B_{k-1} \in \mathbb{R}^{n \times m}$  \\
\textbf{Example:} If acceleration is the control input:
\[
    B_k = \begin{bmatrix} \frac{1}{2} \Delta t^2 \\ \Delta t \end{bmatrix}
\]

\section{Output Matrix ($C_k$)}
\textbf{Symbol:} $C_k \in \mathbb{R}^{r \times n}$, where $r$ is the number of measurements.  \\
\textbf{Example:} If we measure only position:
\[
    C_k = \begin{bmatrix} 1 & 0 \end{bmatrix}
\]

\section{Measurement Noise Vector ($v_k$)}
\textbf{Symbol:} $v_k \in \mathbb{R}^{r}$  \\
\textbf{Covariance Matrix:}
\[
    R_k = E[v_k v_k^T]
\]

\section{Kalman Filter Steps}
\subsection{Prediction Step}
\begin{align*}
    \hat{x}_{k|k-1} &= A_{k-1} \hat{x}_{k-1|k-1} + B_{k-1} u_{k-1} \\
    P_{k|k-1} &= A_{k-1} P_{k-1|k-1} A_{k-1}^T + Q_k
\end{align*}

\subsection{Update Step}
\begin{align*}
    K_k &= P_{k|k-1} C_k^T (C_k P_{k|k-1} C_k^T + R_k)^{-1} \\
    \hat{x}_{k|k} &= \hat{x}_{k|k-1} + K_k (y_k - C_k \hat{x}_{k|k-1}) \\
    P_{k|k} &= (I - K_k C_k) P_{k|k-1}
\end{align*}

\section*{State Estimates in the Kalman Filter}

\subsection*{Predicted State Estimate (\(\hat{x}_{k}^{-}\))}

This is the state prediction based on the motion model before incorporating a new measurement.

It is computed using the state transition equation:
\[
\hat{x}_{k}^{-} = A \hat{x}_{k-1} + B u_k
\]

It represents where we think the system should be, given previous motion.

\subsection*{Updated State Estimate (\(\hat{x}_{k}\))}

This is the corrected state estimate after incorporating the sensor measurement \(y_k\).

It is computed using the Kalman update equation:
\[
\hat{x}_{k} = \hat{x}_{k}^{-} + K_k (y_k - H \hat{x}_{k}^{-})
\]

It represents the best estimate of the state after blending the prediction with the new measurement.










\end{document}
