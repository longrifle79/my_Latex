\documentclass[12pt]{article}
\usepackage[utf8]{inputenc}
\usepackage{amsmath}

\begin{document}

\section*{Gary Hobson}
\section*{MAT260 Module 4}
\section*{May 26, 2025}

\newpage





\section*{7.7- 2}

\[
L_0 = R_0 = M
\]

% \textbf{Encryption:} \\
\textbf{Round 1:}
\[
L_1 = R_0 = M
\]
\[
R_1 = L_0 \oplus f(R_0, K) = M \oplus (M \oplus K) = K
\]

\textbf{Round 2:}
\[
L_2 = R_1 = K
\]
\[
R_2 = L_1 \oplus f(R_1, K) = M \oplus (K \oplus K) = M
\]

So the final ciphertext is:
\[
C = (L_2, R_2) = (K, M)
\]

%\textbf{How Eve Recovers \( M \) and \( K \):} \\
If Eve captures the ciphertext \( C = (K, M) \), then:
\begin{itemize}
    \item The right half of the ciphertext is the original message \( M \).
    \item The left half of the ciphertext is the key \( K \).
\end{itemize}

% \textbf{Conclusion:} \\
% This cipher is completely insecure. Just by looking at the ciphertext, Eve can directly extract both the key and the plaintext.


\newpage
\section*{7.7- 4}

If I understand the question correctly, Bob receives: \\
\textbf{Ciphertext:} \( C_0, C_1 \)

encryption:
\[
C_0 = R
\]
\[
C_1 = L \oplus f(R) \quad \Rightarrow \quad L = C_1 \oplus f(C_0)
\]

Decryption steps are:
\begin{itemize}
    \item Set \( R = C_0 \).
    \item solve \( L = C_1 \oplus f(R) \) using the same function \( f \).
\end{itemize}
This recovers the original \( L \) and \( R \)

\textbf{This is my understanding.}\\
From encryption:
\[
C_0 = R
\]
\[
C_1 = L \oplus f(R)
\]
From decryption:
\[
L = C_1 \oplus f(C_0) = (L \oplus f(R)) \oplus f(R) = L \oplus (f(R) \oplus f(R)) = L \oplus 0 = L
\]
and \( R = C_0 \) is already known. \\
So, both \( L \) and \( R \) are recovered correctly.




\newpage
\section*{7.7- 6}

Alice uses quadruple DES with keys \( K_1 = \text{all 1s} \) and \( K_2 = \text{all 0s} \). Both are known weak keys in DES, meaning:
\[
E_K(E_K(m)) = m
\]
This holds for both \( K_1 \) and \( K_2 \), so Alice's encryption becomes:
\[
c = E_{K_1}(E_{K_1}(E_{K_2}(E_{K_2}(m)))) = m
\]
The result is that the ciphertext is identical to the plaintext. 
Eve does not need a meet in the middle attack because the encryption
 function is effectively the identity the cipher is completely broken with these weak keys.

\newpage
\section*{7.7- 7}


\textbf{Initial Permutation (IP) and its inverse:} \\
These are fixed, linear bit rearrangements complementing bits before or after does not change the property.

\textbf{Expansion (E):} \\
This is a rearrangement and duplication of bits complementing a bit before expansion simply results in the complement of the expanded version.

\textbf{Key Mixing (XOR):} \\
If \( f(R, K) = S(E(R) \oplus K) \), then:
\[
E(\overline{R}) \oplus \overline{K} = \overline{E(R)} \oplus \overline{K} = \overline{E(R) \oplus K}
\]
So the input to the S boxes is complemented, but the S boxes are designed so that for every complemented input, the output is also complemented.

\textbf{Overall Feistel Round:} \\
Each round in DES is:
\[
L_i = R_{i-1}, \quad R_i = L_{i-1} \oplus f(R_{i-1}, K_i)
\]
If you flip all bits of \( L_0 \), \( R_0 \), and each round key, then at each round, the outputs are also flipped this propagates throughout.

Thus, if you encrypt a plaintext \( P \) under a key \( K \) to get ciphertext \( C \), then:
\[
E_K(P) = C \quad \Rightarrow \quad E_{\overline{K}}(\overline{P}) = \overline{C}
\]




\end{document}