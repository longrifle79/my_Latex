\documentclass[12pt]{article}
\usepackage[utf8]{inputenc}
\usepackage{amsmath}
\usepackage{array}

\begin{document}




\section*{Gary Hobson}
\section*{Module 3 Homework} 
\section*{May 19, 2023}
\newpage

\section*{5.4 - 5   }
\[
x_{n+2} = c_0 x_n + c_1 x_{n+1} \pmod{3}
\]
this gives:
\[
1, 1, 0, 2, 2, 0, 1, 1
\]
% We need to find coefficients \( c_0 \) and \( c_1 \) that satisfy the recurrence.

% \textbf{Step 1: Write out the recurrence equations} \\
% From the sequence, each element from the third onward (index 2) satisfies:
% \[
% x_{n+2} = c_0 x_n + c_1 x_{n+1} \pmod{3}
% \]
% Let's write the equations using actual values:

\begin{center}
\begin{tabular}{|c|c|c|c|l|}
\hline
\( n \) & \( x_n \) & \( x_{n+1} \) & \( x_{n+2} \) & Equation \\
\hline
0 & 1 & 1 & 0 & \( 0 = c_0 (1) + c_1 (1) \pmod{3} \) \\
1 & 1 & 0 & 2 & \( 2 = c_0 (1) + c_1 (0) \pmod{3} \) \\
\hline
\end{tabular}
\end{center}

% This gives the system:
\[
c_0 + c_1 \equiv 0 \pmod{3}
\]
\[
c_0 \equiv 2 \pmod{3}
\]

% \textbf{Solve the system} \\
So,
\[
c_0 = 2
\]
Substitute,
\[
2 + c_1 \equiv 0 \pmod{3} \quad \Rightarrow \quad c_1 \equiv -2 \equiv 1 \pmod{3}
\]

%\textbf{Final Answer:} \\
\[
c_0 = 2, \quad c_1 = 1 \pmod{3}
\]

% \textbf{Matrix Form (optional, if required)} \\
% Set up the matrix equation from the first two equations:
% \[
% \begin{bmatrix}
% 1 & 1 \\
% 1 & 0
% \end{bmatrix}
% \begin{bmatrix}
% c_0 \\
% c_1
% \end{bmatrix}
% =
% \begin{bmatrix}
% 0 \\
% 2
% \end{bmatrix}
% \pmod{3}
% \]
% Solving this system (as we did above) gives:
% \[
% c_0 = 2, \quad c_1 = 1
% \]


\newpage

\section*{6.6 - 2}


\[
\det(A) = (1)(1) - (1)(1) = 0
\]
So,
\[
\det(A) \equiv 0 \pmod{26}
\]

% \textbf{Step 2: Check invertibility} \\
If \( \det(A) \equiv 0 \pmod{26} \), then:
\[
\gcd(0, 26) = 26 \neq 1
\]
Since the GCD is not 1, the matrix is not invertible modulo 26.

So, decryption is impossible.

%The matrix is not suitable because its determinant is \( 0 \pmod{26} \), so it is not invertible and cannot be used for decryption in the Hill cipher.



\newpage

\section*{6.6 - 6}

% To recover the encryption matrix \( M \), the attacker selects three plaintext blocks of size 3. These should be linearly independent vectors.

% A simple and effective choice for the plaintext vectors is:
\begin{itemize}
    \item \( \text{aaa} \rightarrow \) vector \( (0, 0, 0) \)
    \item \( \text{baa} \rightarrow \) vector \( (1, 0, 0) \)
    \item \( \text{aba} \rightarrow \) vector \( (0, 1, 0) \)
    \item \( \text{aab} \rightarrow \) vector \( (0, 0, 1) \)
\end{itemize}

The last three are needed to form a \( 3 \times 3 \) identity matrix when stacked:
\[
P = \begin{bmatrix}
1 & 0 & 0 \\
0 & 1 & 0 \\
0 & 0 & 1
\end{bmatrix}
\]

Then, the corresponding ciphertext blocks will become the rows of the matrix \( M \):
\[
C = \begin{bmatrix}
? & ? & ? \\
? & ? & ? \\
? & ? & ?
\end{bmatrix}
\quad \Rightarrow \quad M = C
\]

% \textbf{Summary of Steps} \\
Choose three plaintexts:
\[
\text{"baa", "aba", "aab"} \rightarrow \text{these convert to vectors } (1,0,0), (0,1,0), \text{ and } (0,0,1).
\]
Encrypt each using the unknown Hill cipher. \\
Stack the resulting ciphertexts as rows to obtain the matrix \( M \).





\newpage

\section*{6.6 - 8}



\[
M = \begin{bmatrix}
1 & 2 \\
3 & 4
\end{bmatrix} \pmod{26}
\]
%To find two plaintext vectors \( P_1 \neq P_2 \) that encrypt to the same ciphertext, we solve:
\[
M \cdot P_1 \equiv M \cdot P_2 \pmod{26} \quad \Rightarrow \quad M \cdot (P_1 - P_2) \equiv 0 \pmod{26}
\]
%This means we are looking for a nonzero vector \( v = P_1 - P_2 \) in the null space of \( M \pmod{26} \).

Set \( v = \begin{bmatrix} x \\ y \end{bmatrix} \), and solve:
\[
M \cdot v = \begin{bmatrix}
x + 2y \\
3x + 4y
\end{bmatrix}
\equiv
\begin{bmatrix}
0 \\
0
\end{bmatrix}
\pmod{26}
\]
So,
\[
x \equiv -2y \pmod{26}
\]
Substitute
\[
3(-2y) + 4y = -6y + 4y = -2y \equiv 0 \pmod{26} \quad \Rightarrow \quad y \equiv 0 \pmod{13}
\]
Try \( y = 13 \), then:
\[
x = -2y = -2 \cdot 13 = -26 \equiv 0 \pmod{26}
\]
So:
\[
v = \begin{bmatrix}
0 \\
13
\end{bmatrix}
\]
So, for any plaintext \( P \), \( P + v \) will encrypt to the same ciphertext.

%\textbf{Example:} \\
Let:
\[
P_1 = \begin{bmatrix}
1 \\
2
\end{bmatrix}, \quad
P_2 = \begin{bmatrix}
1 \\
15
\end{bmatrix}
\]
Compute the ciphertexts:
\[
M \cdot P_1 = \begin{bmatrix}
1 \cdot 1 + 2 \cdot 2 \\
3 \cdot 1 + 4 \cdot 2
\end{bmatrix}
= \begin{bmatrix}
1 + 4 \\
3 + 8
\end{bmatrix}
= \begin{bmatrix}
5 \\
11
\end{bmatrix}
\]
\[
M \cdot P_2 = \begin{bmatrix}
1 \cdot 1 + 2 \cdot 15 \\
3 \cdot 1 + 4 \cdot 15
\end{bmatrix}
= \begin{bmatrix}
1 + 30 \\
3 + 60
\end{bmatrix}
= \begin{bmatrix}
31 \\
63
\end{bmatrix}
\equiv
\begin{bmatrix}
5 \\
11
\end{bmatrix}
\pmod{26}
\]

\textbf{So,} \( P_1 = (1, 2) \) and \( P_2 = (1, 15) \) encrypt to the same ciphertext under matrix \( M \pmod{26} \).














\end{document}