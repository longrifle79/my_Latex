\documentclass[12pt]{article}
\usepackage[utf8]{inputenc}
\usepackage{amsmath}
\usepackage{array}

\begin{document}




\section*{Gary Hobson}
\section*{Module 3 Homework} 
\section*{May 19, 2023}
\newpage

\section*{5.4 - 5   }
\[
x_{n+2} = c_0 x_n + c_1 x_{n+1} \pmod{3}
\]
this gives:
\[
1, 1, 0, 2, 2, 0, 1, 1
\]
% We need to find coefficients \( c_0 \) and \( c_1 \) that satisfy the recurrence.

% \textbf{Step 1: Write out the recurrence equations} \\
% From the sequence, each element from the third onward (index 2) satisfies:
% \[
% x_{n+2} = c_0 x_n + c_1 x_{n+1} \pmod{3}
% \]
% Let's write the equations using actual values:

\begin{center}
\begin{tabular}{|c|c|c|c|l|}
\hline
\( n \) & \( x_n \) & \( x_{n+1} \) & \( x_{n+2} \) & Equation \\
\hline
0 & 1 & 1 & 0 & \( 0 = c_0 (1) + c_1 (1) \pmod{3} \) \\
1 & 1 & 0 & 2 & \( 2 = c_0 (1) + c_1 (0) \pmod{3} \) \\
\hline
\end{tabular}
\end{center}

% This gives the system:
\[
c_0 + c_1 \equiv 0 \pmod{3}
\]
\[
c_0 \equiv 2 \pmod{3}
\]

% \textbf{Solve the system} \\
So,
\[
c_0 = 2
\]
Substitute,
\[
2 + c_1 \equiv 0 \pmod{3} \quad \Rightarrow \quad c_1 \equiv -2 \equiv 1 \pmod{3}
\]

%\textbf{Final Answer:} \\
\[
c_0 = 2, \quad c_1 = 1 \pmod{3}
\]

% \textbf{Matrix Form (optional, if required)} \\
% Set up the matrix equation from the first two equations:
% \[
% \begin{bmatrix}
% 1 & 1 \\
% 1 & 0
% \end{bmatrix}
% \begin{bmatrix}
% c_0 \\
% c_1
% \end{bmatrix}
% =
% \begin{bmatrix}
% 0 \\
% 2
% \end{bmatrix}
% \pmod{3}
% \]
% Solving this system (as we did above) gives:
% \[
% c_0 = 2, \quad c_1 = 1
% \]


\newpage

\section*{6.6 - 2}


\newpage

\section*{6.6 - 6}


\newpage

\section*{6.6 - 8}



















\end{document}