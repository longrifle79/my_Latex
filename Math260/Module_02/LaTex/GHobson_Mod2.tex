\documentclass[12pt]{article}
\usepackage[utf8]{inputenc}
\usepackage{amsmath}

\begin{document}


\begin{center}
    \textbf{Gary Hobson}\\
    \textbf{Southern New Hampshire University} \\
    \textbf{May 12, 2024} \\
    \textbf{Module 2: Problem Set} \\
\end{center}


\[
5x + 2 \equiv 3x - 7 \pmod{31}
\]



\begin{itemize}
    \item Subtract \( 3x \) from both sides:
    \[
    (5x - 3x) + 2 \equiv -7 \pmod{31}
    \]
    \[
    2x + 2 \equiv -7 \pmod{31}
    \]

    \item Subtract 2 from both sides:
    \[
    2x \equiv -9 \pmod{31}
    \]

    \item Since \( -9 \mod 31 = 22 \), we rewrite:
    \[
    2x \equiv 22 \pmod{31}
    \]

    \item Multiply both sides by the modular inverse of 2 modulo 31. \\
    The inverse of 2 modulo 31 is 16, because:
    \[
    2 \times 16 = 32 \equiv 1 \pmod{31}
    \]

    So:
    \[
    x \equiv 16 \times 22 = 352 \equiv 11 \pmod{31}
    \]
\end{itemize}

\textbf{Final Answer:}
\[
x \equiv 11 \pmod{31}
\]



\section*{16}

Solve the system of congruences:
\[
\begin{cases}
x \equiv 3 \pmod{5} \\
x \equiv 9 \pmod{11}
\end{cases}
\]

\textbf{Step 1: Verify that the moduli are coprime.} \\
Since 5 and 11 are both prime numbers, they are coprime. This satisfies the condition for applying the Chinese Remainder Theorem, which guarantees a unique solution modulo the product of the moduli.

\textbf{Step 2: Compute the product of the moduli.} \\
\[
N = 5 \times 11 = 55
\]

\textbf{Step 3: Compute the individual terms.} \\
For modulus 5:
\[
N_1 = \frac{N}{5} = \frac{55}{5} = 11
\]
Find the inverse of \( N_1 \) modulo 5, i.e., find \( M_1 \) such that:
\[
11 \times M_1 \equiv 1 \pmod{5}
\]
Since \( 11 \mod 5 = 1 \), we have:
\[
M_1 = 1
\]

For modulus 11:
\[
N_2 = \frac{N}{11} = \frac{55}{11} = 5
\]
Find the inverse of \( N_2 \) modulo 11, i.e., find \( M_2 \) such that:
\[
5 \times M_2 \equiv 1 \pmod{11}
\]
Since \( 5 \times 9 = 45 \equiv 1 \pmod{11} \), we have:
\[
M_2 = 9
\]

\textbf{Step 4: Compute the solution.} \\
\[
x = (3 \times N_1 \times M_1) + (9 \times N_2 \times M_2) \mod N
\]
\[
x = (3 \times 11 \times 1) + (9 \times 5 \times 9) \mod 55
\]
\[
x = 33 + 405 \mod 55
\]
\[
x = 438 \mod 55
\]

\textbf{Step 5: Reduce modulo 55.} \\
\[
438 \div 55 = 7 \text{ with a remainder of } 53
\]
\[
x \equiv 53 \pmod{55}
\]

\textbf{Final Answer:} \\
\[
x \equiv 53 \pmod{55}
\]


\newpage
%

\section*{24}
\[
3, 9, 8, 7, 6, 5, 4, 3, 2, 1
\]

Compute the alternating sum:
\[
3 - 9 + 8 - 7 + 6 - 5 + 4 - 3 + 2 - 1 = -2
\]

Now reduce \( -2 \mod 11 \):
\[
-2 \mod 11 = 9
\]



\end{document}