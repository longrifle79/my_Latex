\documentclass[12pt]{article}
\usepackage[utf8]{inputenc}
\usepackage{amsmath}

\begin{document}

Given: The initial AES key is 128 bits of all 1s. So:
\[
W(0) = W(1) = W(2) = W(3) = \text{FF FF FF FF}
\]

\textbf{(1) Compute \( W(4) \) to \( W(7) \):} \\
\[
W(4) = W(0) \oplus T(W(3))
\]
Using the AES key schedule, this results in:
\[
W(4) = \text{E8 E9 E9 E9}
\]
\[
W(5) = W(1) \oplus W(4) = \text{17 16 16 16}
\]
\[
W(6) = W(2) \oplus W(5) = \text{E8 E9 E9 E9}
\]
\[
W(7) = W(3) \oplus W(6) = \text{17 16 16 16}
\]

So:
\[
W(4) = W(6) = \text{E8 E9 E9 E9}
\]
\[
W(5) = W(7) = \text{17 16 16 16}
\]
Also:
\[
W(5) = \sim W(4) \quad (\text{bitwise complement})
\]

\textbf{(2) Show \( W(10) = W(8) \) and \( W(11) = W(9) \):} \\
From the key schedule:
\[
W(8) = W(4) \oplus W(7)
\]
\[
W(9) = W(5) \oplus W(8)
\]
\[
W(10) = W(6) \oplus W(9)
\]
\[
W(11) = W(7) \oplus W(10)
\]
Since:
\[
W(5) \oplus W(6) = \text{FF FF FF FF},
\]
and XORing the same value twice cancels it out:
\[
W(10) = W(8)
\]
\[
W(11) = W(9)
\]

\textbf{Confirmed.}

\end{document}