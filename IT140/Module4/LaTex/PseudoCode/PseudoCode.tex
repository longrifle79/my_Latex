\documentclass[12pt]{article}
\usepackage[utf8]{inputenc}
\usepackage{listings}
\usepackage{xcolor}

% Define pseudocode style without line numbers or frame
\lstset{
    basicstyle=\ttfamily\small,
    keywordstyle=\color{blue}\bfseries,
    stringstyle=\color{red},
    commentstyle=\color{green!50!black},
    numbers=none, % No line numbers
    frame=none, % No frame
    showspaces=false,
    showstringspaces=false,
    breaklines=true,
    breakatwhitespace=true,
    tabsize=4,
    escapeinside={(*}{*)}, % Allows LaTeX commands inside listings
    morekeywords={PRINT, WHILE, READ, IF, THEN, ELSE, ENDIF, SET, BREAK}
}

\begin{document}

\begin{center}
    Gary Hobson \\
    IT 140 \\
    Module 4-3: Pseudocode Revisited
\end{center}

\textbf{PseudoCode:}

\begin{lstlisting}
PRINT "Welcome to the higher/lower game, Bella!"

// Step 1: Get lower and upper bounds with input validation
WHILE true
    READ lowerBound from user with prompt "Enter the lower bound: "
    READ upperBound from user with prompt "Enter the upper bound: "
    IF lowerBound < upperBound THEN
        BREAK
    ELSE
        PRINT "The lower bound must be less than the upper bound."
    ENDIF
ENDWHILE


// Step 2: Generate a random number between lowerBound and upperBound
SET randomNumber to a random integer between lowerBound and upperBound inclusive

// Step 3: Prompt user to guess a number with input validation and feedback
PRINT "Great, now guess a number between " + lowerBound + " and " + upperBound + ": "
WHILE true
    READ guess from user with prompt ""
    IF guess < lowerBound OR guess > upperBound THEN
        PRINT "Please guess a number between " + lowerBound + " and " + upperBound + "."
    ELSE IF guess < randomNumber THEN
        PRINT "Nope, too low."
        PRINT "Guess another number: "
    ELSE IF guess > randomNumber THEN
        PRINT "Nope, too high."
        PRINT "Guess another number: "
    ELSE
        PRINT "You got it!"
        BREAK
    ENDIF
ENDWHILE
\end{lstlisting}

\end{document}
