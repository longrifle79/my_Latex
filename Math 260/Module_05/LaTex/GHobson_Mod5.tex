\documentclass[12pt]{article}
\usepackage[utf8]{inputenc}
\usepackage{amsmath}

\begin{document}


\section*{Gary Hobson}
\section*{MAT260 Module 5}
\section*{June 2, 2025}
\newpage



\section*{8.5-2}
Given: The initial AES key is 128 bits of all 1s. So:
\[
W(0) = W(1) = W(2) = W(3) = \text{FF FF FF FF}
\]

\textbf{(1) Compute \( W(4) \) to \( W(7) \):} \\
\[
W(4) = W(0) \oplus T(W(3))
\]
Using the AES key schedule, this results in:
\[
W(4) = \text{E8 E9 E9 E9}
\]
\[
W(5) = W(1) \oplus W(4) = \text{17 16 16 16}
\]
\[
W(6) = W(2) \oplus W(5) = \text{E8 E9 E9 E9}
\]
\[
W(7) = W(3) \oplus W(6) = \text{17 16 16 16}
\]

So:
\[
W(4) = W(6) = \text{E8 E9 E9 E9}
\]
\[
W(5) = W(7) = \text{17 16 16 16}
\]
Also:
\[
W(5) = \sim W(4) \quad (\text{bitwise complement})
\]

\textbf{(2) Show \( W(10) = W(8) \) and \( W(11) = W(9) \):} \\
From the key schedule:
\[
W(8) = W(4) \oplus W(7)
\]
\[
W(9) = W(5) \oplus W(8)
\]
\[
W(10) = W(6) \oplus W(9)
\]
\[
W(11) = W(7) \oplus W(10)
\]
Since:
\[
W(5) \oplus W(6) = \text{FF FF FF FF},
\]
and XORing the same value twice cancels it out:
\[
W(10) = W(8)
\]
\[
W(11) = W(9)
\]

\textbf{Confirmed.}

\newpage



\section*{8.5-6}
If my machine can test \( 2^{56} \) keys per second, but the AES keyspace has \( 2^{128} \) keys.

To find how long it will take:
\[
\frac{2^{128}}{2^{56}} = 2^{72} \text{ seconds}
\]

Convert seconds into years:
\[
2^{72} \approx 4.7 \times 10^{21} \text{ seconds}
\]
There are approximately \( 3.15 \times 10^{7} \) seconds in a year:
\[
\frac{4.7 \times 10^{21}}{3.15 \times 10^{7}} \approx 1.49 \times 10^{14} \text{ years}
\]

So, it would take approximately 149 trillion years for the machine to brute force all \( 2^{128} \) AES keys even with a speed of \( 2^{56} \) keys per second. This confirms the astronomical strength of AES-128 against brute force attacks.
\newpage







%*******************************************************************
\section*{9.9-1}

\newpage





%*******************************************************************
\section*{9.9-2}

\newpage





%*******************************************************************
\section*{9.9-3}

\newpage




%*******************************************************************
\section*{9.9-6}

\newpage


\end{document}