\documentclass[12pt]{article}
\usepackage[utf8]{inputenc}
\usepackage{amsmath}

\begin{document}
Gary Hobson\\
MAT 260\\
Southern New Hampshire University\\
May 4, 2025\\
\section*{2)}
A shift cipher encrypts by applying a Caesar shift: 
\[
E(x) = (x + k) \mod 26,
\]
and decryption reverses this 
\[
D(y) = (y - k) \mod 26.
\]
Since the key \( k \) is unknown, we brute-force all 25 non-trivial shifts and identify valid English words.

For the ciphertext ZOMCIH:
\begin{itemize}
    \item Shift by 11 \( \rightarrow \) OLIVES
    \item Shift by 13 \( \rightarrow \) NIGHTY
\end{itemize}

For the ciphertext ZKNGZR:
\begin{itemize}
    \item Shift by 6 \( \rightarrow \) TIGERS
    \item Shift by 19 \( \rightarrow \) ALMOND
\end{itemize}
\newpage

\section*{4)}

\[
D(y) = a^{-1} (y - b) \mod 26
\]
where \( a = 9 \), \( b = 1 \), and \( a^{-1} \) is the modular inverse of 9 modulo 26. Since \( \gcd(9, 26) = 1 \), an inverse exists. Testing small values, we find:
\[
9 \cdot 3 = 27 \equiv 1 \mod 26 \quad \Rightarrow \quad a^{-1} = 3
\]
Now apply the decryption formula to each letter:

\begin{itemize}
    \item J \( \rightarrow 9 \): \quad 
    \( D(9) = 3 (9 - 1) = 3 \cdot 8 = 24 \mod 26 = 24 \rightarrow \text{Y} \)
    \item L \( \rightarrow 11 \): \quad 
    \( D(11) = 3 (11 - 1) = 3 \cdot 10 = 30 \mod 26 = 4 \rightarrow \text{E} \)
    \item H \( \rightarrow 7 \): \quad 
    \( D(7) = 3 (7 - 1) = 3 \cdot 6 = 18 \mod 26 = 18 \rightarrow \text{S} \)
\end{itemize}

Plaintext: YES







\newpage

\section*{31)}
\subsection*{part 1}

If Eve can try \( 2^{64} \) keys per day, then the time to exhaust the full keyspace is:
\[
\frac{2^{128}}{2^{64}} = 2^{64} \text{ days}
\]
\[
= 18,446,744,073,709,551,616 \text{ days}
\]
\subsection*{part 2}
If Alice waits 10 years and then uses a computer 100 times faster, her new rate becomes:
\[
2^{64} \times 100 = 2^{64} \times 10^2 \text{ keys per day}.
\]
However, even waiting 3650 days this rate will finish sooner that using a slower computer.
\[
\frac{2^{128}}{2^{64} \cdot 10^2} = \frac{2^{64}}{10^2} + 3650 \text{ days} 
\]
\[
= 184,467,440,737,099,166 \text{ days}
\]



This duration is not as long. It would be faster to buy in 10 years that it would be to start now with a slower computer.



\end{document}