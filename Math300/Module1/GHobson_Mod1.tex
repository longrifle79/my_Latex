\documentclass{article}
\usepackage{amsmath}

\title{Solutions to Exercises 4.12 and 4.14}
\author{}
\date{}

\begin{document}

\maketitle

\section{Exercise 4.12: More Complex Multiple Regression Models}

\subsection{4.43 First-Order Model}
A first-order linear model assumes a linear relationship between the dependent variable \( y \) and the independent variables.

\subsubsection{(a) Two quantitative independent variables}
\[
E(y) = \beta_0 + \beta_1 x_1 + \beta_2 x_2
\]

\subsubsection{(b) Four quantitative independent variables}
\[
E(y) = \beta_0 + \beta_1 x_1 + \beta_2 x_2 + \beta_3 x_3 + \beta_4 x_4
\]

\subsection{4.44 Second-Order Model}
A second-order model includes quadratic terms and interactions to capture non-linear relationships.

\subsubsection{(a) Two quantitative independent variables}
\[
E(y) = \beta_0 + \beta_1 x_1 + \beta_2 x_2 + \beta_3 x_1^2 + \beta_4 x_2^2 + \beta_5 x_1 x_2
\]

\subsubsection{(b) Three quantitative independent variables}
\[
E(y) = \beta_0 + \beta_1 x_1 + \beta_2 x_2 + \beta_3 x_3 + \beta_4 x_1^2 + \beta_5 x_2^2 + \beta_6 x_3^2 + \beta_7 x_1 x_2 + \beta_8 x_1 x_3 + \beta_9 x_2 x_3
\]

\subsection{4.45 Qualitative Predictors}
Qualitative variables require the use of dummy variables.

\subsubsection{(a) Two levels (A and B)}
Define a dummy variable:
\[
D = \begin{cases} 
    1, & \text{if A} \\
    0, & \text{if B}
\end{cases}
\]
The model is:
\[
E(y) = \beta_0 + \beta_1 D
\]

\subsubsection{(b) Four levels (A, B, C, D)}
Define three dummy variables:
\[
E(y) = \beta_0 + \beta_1 D_1 + \beta_2 D_2 + \beta_3 D_3
\]
where:
- \( D_1 = 1 \) for A, 0 otherwise,
- \( D_2 = 1 \) for B, 0 otherwise,
- \( D_3 = 1 \) for C, 0 otherwise.

\subsection{4.46 and 4.47 Graphing First-Order Models}
These exercises involve graphing first-order regression models. The response surfaces are planes, and when plotted against one independent variable, they form straight lines.

\subsection{4.48 and 4.49 Graphing Second-Order Models}
These models contain squared terms and interaction terms, resulting in curved surfaces when plotted. The interaction term modifies the shape of the graph, making it non-parallel.

\section{Exercise 4.14: Regression Model for Oil Removal}
This problem involves fitting a first-order regression model with interaction terms for predicting voltage in an oil removal experiment.

\subsection{Given Model}
\[
E(y) = \beta_0 + \beta_1 x_1 + \beta_2 x_2 + \beta_3 x_5 + \beta_4 x_1 x_2 + \beta_5 x_1 x_5
\]

\subsection{(a) Interaction Effects}
The interaction terms \( x_1 x_2 \) and \( x_1 x_5 \) indicate that the effect of disperse phase volume on voltage is influenced by salinity and surfactant concentration.

\subsection{(b) Model Fitting in Minitab}
Regression analysis should be performed using **Minitab**. The dataset used is the WATEROIL dataset. The results should include:
- Regression coefficients,
- \( R^2 \) and \( R^2_{\text{adj}} \),
- Residual analysis.

\subsection{(c) Interpretation of Coefficients}
- **Main effects** (\(\beta_1, \beta_2, \beta_3\)) show the independent influence of each predictor.
- **Interaction terms** (\(\beta_4, \beta_5\)) indicate how predictors interact.

\subsection{Minitab Analysis}
\begin{enumerate}
    \item Load the WATEROIL dataset into Minitab.
    \item Run multiple regression:
    \[
    y = \beta_0 + \beta_1 x_1 + \beta_2 x_2 + \beta_3 x_5 + \beta_4 x_1 x_2 + \beta_5 x_1 x_5
    \]
    \item Extract:
    \begin{itemize}
        \item Coefficient estimates,
        \item \( R^2 \) for model fit,
        \item Residuals to check assumptions.
    \end{itemize}
\end{enumerate}

\subsection{Conclusion}
If \( R^2 \) is high and p-values for interactions are low, the model is a strong predictor. Otherwise, model improvements are needed.

\end{document}
